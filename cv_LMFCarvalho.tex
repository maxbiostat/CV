% !TEX program = pdflatex
\documentclass[10pt]{article}
\usepackage{array, xcolor, indentfirst, hyperref}
\usepackage{splitbib, booktabs} %% ATTENTION: must be loaded AFTER everything, due to redefining \bibitem and \thebibliography
\usepackage[affil-it]{authblk}
\usepackage[margin=3cm]{geometry}
%\title{\bfseries Curriculum Vit\ae}
\title{\bfseries Luiz Max Fagundes de Carvalho}
\author{\href{mailto:lmax.procc@gmail.com}{\nolinkurl{lmax.procc@gmail.com}}}
\affil{Programme for Scientific Computing (PROCC) and National School of Public Health (ENSP), Oswaldo Cruz Foundation, Brazil.}
\date{Last update: November 2018}
%%% tweaks
\renewcommand\refname{Publications} 
\definecolor{lightgray}{gray}{0.8}
\newcolumntype{L}{>{\raggedleft}p{0.14\textwidth}}
\newcolumntype{R}{p{0.8\textwidth}}
\newcommand\VRule{\color{lightgray}\vrule width 0.5pt}
%%% end of tweaks
\begin{document}
\maketitle

\section*{Summary}

Natural biological processes emit signals, which are often too loud or too low for us to ``hear''.
My goal as a scientist is to develop and apply statistical and mathematical tools to decode and quantify these biological signals, specially those imprinted in pathogens' -- and their hosts' -- genomes.
I hope a better understanding of these entities can contribute to the general theory of Biology and also lead to a progressive reduction of the world's disease burden.

My interests lie in \textbf{quantitative Biology} and \textbf{Biostatistics}, ranging from complex networks to spatial analysis to statistical phylogenetics.
In recent years my main scientific interest has been to understand the complex interactions between rapidly evolving pathogens, such as {RNA} viruses, and their hosts.
My peer-reviewed papers have been published in indexed international journals, such as \textit{Infection, Genetics and Evolution}, \textit{Transactions of the Royal Society of Tropical Medicine and Hygiene} and \textit{BMC Bioinformatics} (\textit{Nature}, \textit{Science} and \textit{Cell} too, but don't get too excited about those).

I have also attended or had work presented at national and international conferences, such as the Brazilian Congress of Virology (BSV), the Gordon Research Conference on Biology of Host-Parasite Interactions, the World Statistics Congress (ISI), the Joint Statistical Meetings (ASA) and, my favourite, the Brazilian Meeting on Bayesian Statistics (ISBra--EBEB).

As you will notice if you continue reading, I am a big fan of collaboration, interacting with colleagues around Brazil and abroad.
My current interests are:
\begin{itemize}
\itemsep0.1em
 \item [-] Phylogeny estimation: MCMC exploration of time-tree space -- characterising time-tree space, new transition kernels;
 \item [-] Coupling mathematical models to coalescent-based population reconstructions;
 \item [-] Bayesian inference of deterministic models;
 \item [-] Combining (pooling) probability distributions;
\end{itemize}

Please feel free to contact me if your interests lie anywhere near these topics.

\begin{itemize}
\itemsep0.1em
 \item[] Linkedin: \url{http://www.linkedin.com/profile/view?id=171872451}\\
 \item[] ResearchGate: \url{https://www.researchgate.net/profile/Luiz_Carvalho11}\\
 \item[] Lattes CV: \url{ http://lattes.cnpq.br/7282202947621572} 
\end{itemize}

\newpage
A list of PDFs of my publications can be found at \url{https://github.com/maxbiostat/papers/tree/master/PAPERS}
\nocite{*}
\begin{category}{Published/Accepted -- peer reviewed}
\SBentries{Camara2011,Carvalho2012,Camara2013,Carvalho2013,Buss2014,Mir2014,Coelho2015,Rambaut2016a,Rambaut2016b,Codeco2016,Diehl2016,Coelho2016,Villela2017,Dudas2017,Dudas2018}
\end{category}

\begin{category}{Conference papers}
\SBentries{Carvalho2011}
\end{category}

\begin{category}{Book chapters}
\SBentries{Carvalho2014chap,CamaraCarvalho2014,Carvalho_etal_2015A}
\end{category}

\begin{category}{Submitted}
\SBentries{}
\end{category}
\bibliographystyle{ieeetr}
\bibliography{publication}
\section*{Work in progress\footnote{Drafts in final phase of preparation}}

\underline{Carvalho, L.M.}, G.~Baele, N.~Faria, A.~M. Perez, M.~Suchard, P.~Lemey, and  W.~C. Silveira, ``{S}patiotemporal {D}ynamics of foot-and-mouth disease virus in {S}outh {A}merica'' In preparation.

L.~Zimmermann, \underline{Carvalho, L.M.}, L.~Vasconcellos, L.~Bastos, C.~Struchiner, and A.~H. Lopes, ``{T}emperature-dependent oviposition and egg eclosion of {C}hagas disease vector \textit{{R}hodnius prolixus}'', In preparation.

\underline{Carvalho, L.M.}, D.A.~Vilella, F.C.~Coelho, L.~Bastos, ``{O}n the choice of the weights for the logarithmic pooling of probability distributions'', In preparation.

\underline{Carvalho, L.M.}, G.~Baele, M.A.~Suchard, A.~Rambaut, ``{A}n efficient, tunable time-tree transition kernel for Bayesian phylogenetics'', In preparation.

\newpage
\section*{Education}
\begin{tabular}{L!{\VRule}R}
2009--2012&BSc Microbiology and Immunology, Federal University of Rio de Janeiro, Brazil.\\
2014--2018& PhD Evolutionary Biology, University of Edinburgh, UK.
\end{tabular}

\section*{Professional Experience}
\begin{tabular}{L!{\VRule}R}
2010--2013&{
Pan American Health Organization (PAHO)\newline
Position: Statistical Assistant\newline
Role: Developed and analysed quality control experiments for veterinary diagnostic tests;\newline
Research on Foot-and-Mouth Disease virus phylodynamics 
}
\end{tabular}
\section*{Academic Experience}
\begin{tabular}{L!{\VRule}R}
2009--2011&{
Sector of Infectious Diseases Epidemiology (SEDI), Institute of Microbiology, Federal University of Rio de Janeiro\newline
Position: Scientific initiation student\newline
Supervisor: Prof. Dr. Fernando Portela C\^amara\newline
Role: research on statistical methods in the epidemiology of AIDS, sylvatic yellow fever and dengue
}\\
\\
2012--2013&{Programme for Scientific Computing (PROCC), Oswaldo Cruz Foundation (Fiocruz)\newline
Position: Scientific initiation student\newline
Supervisor: Prof. Dr. Oswaldo Gon\c{c}alves Cruz\newline
Role: research on spatial partition methods for health areal data
}\\
\\
2013--2014&{Programme for Scientific Computing (PROCC), Oswaldo Cruz Foundation (Fiocruz)\newline
Position: Scientific initiation student\newline
Supervisors: Prof. Dr. Claudio Struchiner and Dr. Leonardo Bastos\newline
Role: research on Bayesian inference of deterministic population growth models, multilevel binary regression and opinion pooling
}\\
\\
2014-- 2018& {Institute of Evolutionary Biology (SBS), University of Edinburgh\newline
Position: PhD student\newline
Supervisors: Andrew Rambaut and Darren Obbard\newline
Role: research on statistical phylogenetics methods for RNA virus phylodynamics.
}\\
\\
2018--& {Programme for Scientific Computing (PROCC) and National School of Public Health (ENSP), Oswaldo Cruz Foundation\newline
Position: Postdoctoral Researcher\newline
Supervisor: Claudio Struchiner\newline
Role: research on statistical methods applied to Public Health.
}\\
\end{tabular}

\newpage
\subsection*{Conferences\footnote{Main conferences I have attended to or had work presented at. For a complete list please visit my Lattes CV.}}
\subsubsection*{Presented work}
\begin{tabular}{L!{\VRule}R}
2009&{Gomes, A.L.B.B; \underline{Carvalho, L.M.}; Camara, F.P. \textbf{Din\^amica Espacial da Dengue no Rio de Janeiro: 1986 a 2009 [Spatial dynamics of dengue in Rio de Janeiro: 1986 to 2009]}, 2009,  II International Congress on Geography of Health, Uberl\^andia -MG, Brazil.}\\
\\
2011&{Vianez Jr., J.L.; \underline{Carvalho, L. M.}; Bisch, P. \textbf{Development of a workflow for large-scale epitope prediction: dengue virus as a study of case}. In: X Meeting 2011, 2011, Florianópolis--SC, Brazil. X Meeting 2011 Abstract Book, 2011. v. ID 234.}\\
\\
2011&{\underline{Carvalho,L.M.}; Santos, L.B.; Silveira, W.C. \textbf{Phylodynamics of Foot-and-Mouth Disease Virus: a Complex Network approach}. XXII Meeting of the Brazilian Society of Virology, \'Aguas de Lind\'oia -- SP, Brazil.}\\
\\
2012&{\underline{Carvalho, L. M.}; Santos, L.B.; Faria, N.R.; Silveira, W.C. \textbf{Phylogeographic Dynamics of foot-and-mouth disease virus in Ecuador 2002 to 2010}, 2012 In 17th International BioInformatics Workshop on Virus Evolution and Molecular Epidemiology, Belgrade, Serbia.}\\
\\
2012&{\underline{Carvalho, L.M.}; Faria, N.R.; Silveira, W.C. \textbf{Phylodynamics of Foot-and-Mouth Disease Virus in South America: a Comprehensive Analysis}. XXIII Brazilian Congress of Virology, Foz do Igua\c{c}u -- PR, Brazil.}\\
\\
2012&{Vasconcellos, L.R.C.; Dias, F.A.; Soares, J. B. R. C.; \underline{Carvalho, L. M.}; Oliveira, M. M.; Alves e Silva, T. L.; Gon\c{c}alves, I.; Oliveira, M. F.; Lopes, F.G.; Lopes, A. H. C. S. \textbf{Interaction of the hemipteran Oncopeltus fasciatus with the trypanosomatid Leptomonas wallacei: an insight into parasitism}. In: Gordon Research Conference on Biology of Host-Parasite Interactions, 2012, Newport, RI, USA. Annals of the Gordon Research Conference on Biology of Host-Parasite Interactions, 2012. v. 1. p. 1-2.}\\
\\
2013&{Zimmermann, L.T.; \underline{Carvalho, L.M.};  Vasconcellos, L.R.; Bastos, L.S.; Struchiner, C.J.; Lopes, A.H. \textbf{{T}emperature-dependent oviposition and egg eclosion of {C}hagas disease vector \textit{{R}hodnius prolixus}} In XXIX Annual Meeting of the Brazilian Society of Protozoology, 2013, Caxambu-MG, Brazil. Abstract book of the XXIX Annual Meeting of the SBPz, p. 159, ID V004.}\\
\\
2014&{\underline{Carvalho, L. M.}; Struchiner, C.J.; Bastos, L.S. \textbf{Bayesian inference of deterministic population growth models} In XII Brazilian Meeting on Bayesian Statistics (EBEB), 2014, Atibaia-SP, Brazil. Abstract book of the XII Brazilian Meeting on Bayesian Statistics, p. 63}\\
\\
2014&{Bastos, L. S.; \underline{Carvalho, L. M.} \textbf{Random Effects Binary Model with Misclassified Response} In Joint Statistical Meetings, 2014, Boston MS, USA.}\\
\end{tabular}
\subsubsection*{Participated}
\begin{tabular}{L!{\VRule}R}
2012&{VIII Brazilian Congress of Epidemiology, S\~ao Paulo--SP, Brazil.}\\
\\
2013&{1\textsuperscript{st} Symposium on Big Data and Public Health, Rio de Janeiro--RJ, Brazil.}
\end{tabular}
\subsection*{Invited Talks}
\begin{tabular}{L!{\VRule}R}
2011&{\textit{Phylodynamics of Foot-and-Mouth Disease Virus: a Complex Network approach}. XXII Meeting of the Brazilian Society of Virology}\\
\\
2011&{\textit{Playing Dumb: The Misuse of Statistics in Biology}. Institute of Microbiology, Federal University of Rio de Janeiro.}\\
\\
2012&{\textit{Knowledge Discovery in Databases through Complex Networks: application to phylodynamics}. WaFIS 2012}\\
\\
2014&{\textit{Bayesian inference of deterministic population growth models}. XII Brazilian Meeting on Bayesian Statistics}\\
\\
2015&{\textit{Phylodynamics: how Genetics and Mathematics are changing our understanding of infectious diseases}. 30rd Brazilian Mathematics Colloquium}\\
\\
2015&{\textit{Choosing the weights for the logarithmic pooling of probability distributions}. 60th World Statistics Congress.}\\
\\
2016&{\textit{On the choice of the weights for the logarithmic pooling of probability distributions}. XIII Brazilian Meeting on Bayesian Statistics.}\\
\\
\end{tabular}
\subsection*{Memberships}
Brazilian Society for Virology (SBV), Brazilian Statistical Association (ABE), Brazilian Society for the Advancement of Science (SBPC).
\subsection*{Teaching Experience}
\begin{tabular}{L!{\VRule}R}
2007--2011&{
\textbf{High School Chemistry and Biology}\newline
I was a voluntary teacher of whole-year high school courses on organic chemistry, general chemistry and biology.
}\\
2010--2013&{
\textbf{Basics of Mathematics and Statistics for Microbiology}\newline
Federal University of Rio de Janeiro\newline
Supervisor: Prof. Dr. Fernando Portela C\^amara\newline
Basics on descriptive statistics, Gaussian distribution and hypothesis testing.
Lately, some basic calculus too.
}\\
2010&{
\textbf{Topics in Human Physiology}\newline
Federal University of Rio de Janeiro\newline
Supervisor: Prof. Dr. Pedro Paulo Elsas\newline
By means of seminars and group discussions, we discussed particular aspects of human physiology and stimulate the students to draw general conclusions about the subjacent biological processes going on.
}\\
2012&{
\textbf{Bioinformatics}\newline
Federal University of Rio de Janeiro\newline
Supervisor: Prof. Andrew Macrae, PhD \newline
Basics on Bioinformatics: basic genome annotation, databases, alignment, phylogenetics.
}\\
2014-2017&{
\textbf{Molecular Evolution}\newline
University of Edinburgh\newline
Supervisor: Prof. Andrew Rambaut, PhD \newline
Molecular phylogenetics.
}\\
2017&{
\textbf{Statistics for Genetics}\newline
University of Edinburgh\newline
Supervisor: Ian White\newline
TA in the Bayesian module
}\\
\end{tabular}
\subsection*{Awards}
\begin{tabular}{L!{\VRule}R}
2010&{Honourable Mention -  XVI Week of  Microbiology and Immunology, Federal University of Rio de Janeiro.}\\
2011&{Honourable Mention -  XVII Week of  Microbiology and Immunology, Federal University of Rio de Janeiro.}\\
2011&{Selected for Oral presentation -- XXII  National Meeting of the Brazilian Society for  Virology.}\\
2012&{Honourable Mention -  XVIII Week of  Microbiology and Immunology.}\\
2014&{Selected for Oral presentation -- XII  Brazilian Meeting on Bayesian  Statistics.}\\
2014& {Principal's Career Development Scholarship, University of Edinburgh.}\\
\end{tabular}
\section*{Languages}
\begin{tabular}{L!{\VRule}R}
Portuguese & Native\\
English& Fluent (CAE -- Grade A)\\
Spanish& Advanced\\
\end{tabular}
\section*{References}
\begin{tabular}{lcl}%{L!{\VRule}R}
\toprule
Reference & What for & email \\
\midrule
Prof. Dr. Fernando Portela C\^amara&Research and Teaching & \href{mailto:portela@micro.ufrj.br}{\nolinkurl{portela@micro.ufrj.br}}\\
\\
Prof. \^Angela Hampshire Lopes, PhD &Research & \href{mailto:angela.lopes@micro.ufrj.br}{\nolinkurl{angela.lopes@micro.ufrj.br}}\\
\\
Prof. Andrew Macrae, PhD&Teaching & \href{mailto:amacrae@biologia.ufrj.br}{\nolinkurl{amacrae@biologia.ufrj.br}}\\
\\
Prof. Dr. Pedro Paulo Xavier Elsas &Teaching& \href{mailto:pxelsas@micro.ufrj.br}{\nolinkurl{pxelsas@micro.ufrj.br}}\\
\\
Prof. Claudio Struchiner, PhD &Research&  \href{mailto:stru@fiocruz.br}{\nolinkurl{stru@fiocruz.br}} \\
\\
Leonardo Bastos, PhD &Research&  \href{mailto:lsbastos@fiocruz.br}{\nolinkurl{lsbastos@fiocruz.br}} \\
\\
Prof. Philippe Lemey, PhD & Research& \href{mailto:philippe.lemey@rega.kuleuven.be}{\nolinkurl{philippe.lemey@rega.kuleuven.be}}\\
\\
Prof. Flavio Code\c{c}o Coelho, PhD & Research & \href{mailto:fccoelho@fgv.br}{\nolinkurl{fccoelho@fgv.br}}\\
\bottomrule
\end{tabular}
\end{document}
